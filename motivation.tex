\documentclass{article}
\usepackage[toc,page]{appendix}
\usepackage{blindtext}
\usepackage{titlesec}
\usepackage{graphicx} % Required for inserting images
\usepackage{float}
\usepackage{listings}
\usepackage{amsmath}
\usepackage[a4paper, total={5in, 8in}]{geometry}
\usepackage{hyperref}
\usepackage{biblatex}
\addbibresource{ref.bib}
\hypersetup{
    colorlinks=true,
    linkcolor=blue,
    filecolor=magenta,      
    urlcolor=cyan,
    pdftitle={Overleaf Example},
    pdfpagemode=FullScreen
}

\setlength{\parindent}{0pt} % No indentation
\title{Motivation for thesis project}

\author{Sune Skaanning Engtorp \\ fpm268@alumni.ku.dk}
\date{\today}

\begin{document}

\maketitle

Structure editors provide a way to manipulate some program's abstract syntax structure directly, in contrast to writing and editing a program's source code in plain-text. An early example of a proposed structure edtior is the Cornell Program Synthesizer\cite{cornell} from 1981.

However structure editors like the Cornell Program Synthesizer\cite{cornell} allow the programmer to introduce syntactically ill-formed programs. This problem has been studied by Omar et al. who introduced Hazel\cite{omar}, a programming environment for an Elm-like functional language with typed holes, which allows for evaluation to continue past holes which might be empty or ill-formed. The Hazel environment is based on the Hazelnut structure editor calculus that allows for finite edit expressions and inserts holes automatically to guarantee that every editor state has some type.

With inspiration from Hazel, Godiksen et. al \cite{godiksen} presented a type-safe structure editor calculus which manipulates a simply-typed lambda calculus. The editor calculus ensures that if an edit action is well-typed, then the resulting program is also well-typed. The editor calculus and programming language has later been used to implement a type-safe structure editor in Elm\cite{PAINT2023-missing ref}.

A common property of the editor calculi and editors mentioned until now is that they are built to work with only one programming language. The calculi are strongly dependent on the language they can manipulate, and if the language were to change, it could require re-writing the complete editor calculus.

To tackle this problem, one can draw inspiration from the Centaur system\cite{centaur} that is able to take a language syntax described in the Metal language as input and generate a structure-oriented editor. However the system lacks a dedicated type-safe editor calculus. Such a type-safe generalized editor calculus has been proposed by \cite{missing ref}.

The aim for this project is to implement parts of a type-safe generalized structure editor based on the proposed generalized editor calculus, with emphasis on generality. A requirement for a successful implementation is the editor being able to take the syntax of any language and generate 


\printbibliography

\end{document}