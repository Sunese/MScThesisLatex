\section{Conclusion}
This project was exploratory in nature, aiming to implement a generic editor
based on the work of the Aalborg project\cite{aalborg} and other related work.

We have shown that it is possible to implement a program able to instantiate
an editor for any language, given its abstract syntax.

Only atomic prefix commands have been implemented, including the ability to
move the cursor and perform substitutions. This was a necessary first step
to implement the remaining editor expressions, which unfortunately were not
completed. This however has led to a minimal viable product, which is a
program that can instantiate an editor for any language, where basic
edits can be performed on the syntax tree directly.

\subsection{Future work}
Editor expressions for the remaining commands should be implemented, this is possible by following a similar approach to the one used for the atomic prefix commands, i.e. by implementing function generators for each command based on its semantics.

Given the generic nature of the editor, it would be interesting to see
an implementation in a language with support for typeclasses, such as Haskell.
This should allow for a more concise implementation overall, being less rigid
than the current implementation, since it relies on references to functions and
data types via a low-level interface in the Elm CodeGen package, which refers
to them by direct string manipulation. This is in contrast to typeclass
instances, which would allow for a more abstract representation of a tree
of any sort.


