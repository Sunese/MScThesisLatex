\documentclass{article}
\usepackage[toc,page]{appendix}
\usepackage{blindtext}
\usepackage{titlesec}
\usepackage{graphicx} % Required for inserting images
\usepackage{float}
\usepackage[ddmmyyyy,hhmmss]{datetime}
\usepackage{listings}% http://ctan.org/pkg/listings
\usepackage{amsmath}
\usepackage[a4paper, total={6in, 10in}]{geometry}
\usepackage{hyperref}
\usepackage{biblatex}
\usepackage[dvipsnames]{xcolor}
\usepackage{parskip}
\usepackage{luacolor} % Required to use the lua-ul \highLight command 
\usepackage{lua-ul} 
\usepackage{lmodern}
\usepackage[most]{tcolorbox}
\usepackage{varwidth}   %% provides varwidth environment

\colorlet{mycolor}{teal!80!blue}
\tcbset{
    enhanced,
    colback=mycolor!5!white,
    boxrule=0.1pt,
    colframe=mycolor!80!white,
    fonttitle=\bfseries
}

\newtcolorbox[blend into=figures]{myfigure}[3][]{title={#2},#1,hbox,label={#3}}

\addbibresource{references/ref.bib}

\hypersetup{
    colorlinks=true,
    linkcolor=cyan,
    citecolor=cyan,
    filecolor=magenta,      
    urlcolor=cyan,
    pdftitle={Overleaf Example},
    pdfpagemode=FullScreen
}
\LuaULSetHighLightColor{yellow}

\definecolor{codegreen}{rgb}{0,0.6,0}
\definecolor{codegray}{rgb}{0.5,0.5,0.5}
\definecolor{codepurple}{rgb}{0.58,0,0.82}
\definecolor{backcolour}{rgb}{0.95,0.95,0.92}

\lstdefinestyle{mystyle}{
    backgroundcolor=\color{backcolour},   
    commentstyle=\color{codegreen},
    keywordstyle=\color{magenta},
    numberstyle=\tiny\color{codegray},
    stringstyle=\color{codepurple},
    basicstyle=\ttfamily\footnotesize,
    breakatwhitespace=false,         
    breaklines=true,                 
    captionpos=b,                    
    keepspaces=true,                 
    numbers=left,                    
    numbersep=5pt,                  
    showspaces=false,                
    showstringspaces=false,
    showtabs=false,                  
    tabsize=2
}

\lstset{style=mystyle}

\setlength{\parindent}{0pt} % No indentation
\title{The generalized editor calculus \\ Kommentarer og spørgsmål}
\date{\today}
\author{Sune Skaanning Engtorp}

\begin{document}
\maketitle

\begin{itemize}
    \item Side 14
        \begin{itemize}
            \item Ifølge definition 13 tilføjer vi kun nye operatorer til $\mathcal{O}^C$ fra $\hat{\mathcal{O}}$ hvis de har aritet $( \overrightarrow{\hat{s}_1}.\hat{s}_1,...,\overrightarrow{\hat{s}_n}.\hat{s}_n)\hat{s}_n$ hvor $1\leq i \leq n$. Dette burde derfor også gælde $exp$ operatoren med aritet $(e)s$, men jeg kan ikke se den i figur 8. Spørger om dette for at være sikker på jeg ikke misforstår definition 13.

            \item Der står at $C$ ikke kan indeholde nogen cursor. Jeg kan ikke se hvor dette er forklaret og hvorfor det er nyttigt.
        \end{itemize}
    \item Side 20
        \begin{itemize}
            \item Når jeg kigger på figur 20 og 21, kan jeg ikke udlede hvordan operatorer uden bindere er blevet indkodet. Figur 20 bruger lambda-funktioner med bindere som argument, men hvad hvis der ikke er nogen bindere?
        \end{itemize}
    \item Side 21
        \begin{itemize}
            \item Dette er ikke super vigtigt, men har bemærket at deres eksempel på $let$ operatoren ser ud til altid at binde en $x$ variabel. Er der en speciel grund til det?
        \end{itemize}
    \item Side 24
        \begin{itemize}
            \item Jeg synes figur 28 er meget svær at forstå og håber vi kan gennemgå den sammen fra bunden.
        \end{itemize}
    \item Side 25
        \begin{itemize}
            \item Som jeg skriver, giver de her funktioner rigtig god mening, med min egen antagelse af hvordan pattern matching virker.
        \end{itemize}
\end{itemize}



\printbibliography

\end{document}